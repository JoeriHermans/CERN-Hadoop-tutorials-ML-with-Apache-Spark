% ****** Start of file aipsamp.tex ******
%
%   This file is part of the AIP files in the AIP distribution for REVTeX 4.
%   Version 4.1 of REVTeX, October 2009
%
%   Copyright (c) 2009 American Institute of Physics.
%
%   See the AIP README file for restrictions and more information.
%
% TeX'ing this file requires that you have AMS-LaTeX 2.0 installed
% as well as the rest of the prerequisites for REVTeX 4.1
%
% It also requires running BibTeX. The commands are as follows:
%
%  1)  latex  aipsamp
%  2)  bibtex aipsamp
%  3)  latex  aipsamp
%  4)  latex  aipsamp
%
% Use this file as a source of example code for your aip document.
% Use the file aiptemplate.tex as a template for your document.
\documentclass[%
 aip,
 jmp,%
 amsmath,amssymb,
%preprint,%
 reprint,%
%author-year,%
%author-numerical,%
]{revtex4-1}

\usepackage{graphicx}% Include figure files
\usepackage{dcolumn}% Align table columns on decimal point
\usepackage{bm}% bold math
%\usepackage[mathlines]{lineno}% Enable numbering of text and display math
%\linenumbers\relax % Commence numbering lines

\begin{document}

%\preprint{AIP/123-QED}

\title[Analytics with Apache Spark and MLlib tutorial notes]{Analytics with Apache Spark and MLlib tutorial notes}% Force line breaks with \\

\author{A. Antonio R. Marin}%
 \email{antonio.romero.marin@cern.ch}
 \affiliation{CERN, IT-DB-SAS}
\author{B. Joeri R. Hermans}
 \email{joeri.hermans@cern.ch}
 \affiliation{CERN, IT-DB-SAS}

\date{\today}% It is always \today, today,
             %  but any date may be explicitly specified

\begin{abstract}
An article usually includes an abstract, a concise summary of the work
covered at length in the main body of the article. It is used for
secondary publications and for information retrieval purposes.
%
Valid PACS numbers may be entered using the \verb+\pacs{#1}+ command.
\end{abstract}
\keywords{Apache Spark, MLlib, Analytics, Machine Learning, Distributed Computing}%Use showkeys class option if keyword
                              %display desired
\maketitle

\section{\label{sec:level1}First-level heading}

This sample document demonstrates proper use of REV\TeX~4.1 (and
\LaTeXe) in manuscripts prepared for submission to AIP
journals. Further information can be found in the documentation included in the distribution or available at
\url{http://authors.aip.org} and in the documentation for
REV\TeX~4.1 itself.

When commands are referred to in this example file, they are always
shown with their required arguments, using normal \TeX{} format. In
this format, \verb+#1+, \verb+#2+, etc. stand for required
author-supplied arguments to commands. For example, in
\verb+\section{#1}+ the \verb+#1+ stands for the title text of the
author's section heading, and in \verb+\title{#1}+ the \verb+#1+
stands for the title text of the paper.

Line breaks in section headings at all levels can be introduced using
\textbackslash\textbackslash. A blank input line tells \TeX\ that the
paragraph has ended.

\subsection{\label{sec:level2}Second-level heading: Formatting}

This file may be formatted in both the \texttt{preprint} (the default) and
\texttt{reprint} styles; the latter format may be used to
mimic final journal output. Either format may be used for submission
purposes; however, for peer review and production, AIP will format the
article using the \texttt{preprint} class option. Hence, it is
essential that authors check that their manuscripts format acceptably
under \texttt{preprint}. Manuscripts submitted to AIP that do not
format correctly under the \texttt{preprint} option may be delayed in
both the editorial and production processes.

The \texttt{widetext} environment will make the text the width of the
full page, as on page~\pageref{eq:wideeq}. (Note the use the
\verb+\pageref{#1}+ to get the page number right automatically.) The
width-changing commands only take effect in \texttt{twocolumn}
formatting. It has no effect if \texttt{preprint} formatting is chosen
instead.

\subsubsection{\label{sec:level3}Third-level heading: Citations and Footnotes}

Citations in text refer to entries in the Bibliography;
they use the commands \verb+\cite{#1}+ or \verb+\onlinecite{#1}+.
Because REV\TeX\ uses the \verb+natbib+ package of Patrick Daly,
its entire repertoire of commands are available in your document;
see the \verb+natbib+ documentation for further details.
The argument of \verb+\cite+ is a comma-separated list of \emph{keys};
a key may consist of letters and numerals.

By default, citations are numerical; \cite{feyn54} author-year citations are an option.
To give a textual citation, use \verb+\onlinecite{#1}+: (Refs.~\onlinecite{witten2001,epr,Bire82}).
REV\TeX\ ``collapses'' lists of consecutive numerical citations when appropriate.
REV\TeX\ provides the ability to properly punctuate textual citations in author-year style;
this facility works correctly with numerical citations only with \texttt{natbib}'s compress option turned off.
To illustrate, we cite several together \cite{feyn54,witten2001,epr,Berman1983},
and once again (Refs.~\onlinecite{epr,feyn54,Bire82,Berman1983}).
Note that, when numerical citations are used, the references were sorted into the same order they appear in the bibliography.

A reference within the bibliography is specified with a \verb+\bibitem{#1}+ command,
where the argument is the citation key mentioned above.
\verb+\bibitem{#1}+ commands may be crafted by hand or, preferably,
generated by using Bib\TeX.
The AIP styles for REV\TeX~4 include Bib\TeX\ style files
\verb+aipnum.bst+ and \verb+aipauth.bst+, appropriate for
numbered and author-year bibliographies,
respectively.
REV\TeX~4 will automatically choose the style appropriate for
the document's selected class options: the default is numerical, and
you obtain the author-year style by specifying a class option of \verb+author-year+.

This sample file demonstrates a simple use of Bib\TeX\
via a \verb+\bibliography+ command referencing the \verb+aipsamp.bib+ file.
Running Bib\TeX\ (in this case \texttt{bibtex
aipsamp}) after the first pass of \LaTeX\ produces the file
\verb+aipsamp.bbl+ which contains the automatically formatted
\verb+\bibitem+ commands (including extra markup information via
\verb+\bibinfo+ commands). If not using Bib\TeX, the
\verb+thebibiliography+ environment should be used instead.

\paragraph{Fourth-level heading is run in.}%
Footnotes are produced using the \verb+\footnote{#1}+ command.
Numerical style citations put footnotes into the
bibliography\footnote{Automatically placing footnotes into the bibliography requires using BibTeX to compile the bibliography.}.
Author-year and numerical author-year citation styles (each for its own reason) cannot use this method.
Note: due to the method used to place footnotes in the bibliography, \emph{you
must re-run BibTeX every time you change any of your document's
footnotes}.

\section{Math and Equations}
Inline math may be typeset using the \verb+$+ delimiters. Bold math
symbols may be achieved using the \verb+bm+ package and the
\verb+\bm{#1}+ command it supplies. For instance, a bold $\alpha$ can
be typeset as \verb+$\bm{\alpha}$+ giving $\bm{\alpha}$. Fraktur and
Blackboard (or open face or double struck) characters should be
typeset using the \verb+\mathfrak{#1}+ and \verb+\mathbb{#1}+ commands
respectively. Both are supplied by the \texttt{amssymb} package. For
example, \verb+$\mathbb{R}$+ gives $\mathbb{R}$ and
\verb+$\mathfrak{G}$+ gives $\mathfrak{G}$

In \LaTeX\ there are many different ways to display equations, and a
few preferred ways are noted below. Displayed math will center by
default. Use the class option \verb+fleqn+ to flush equations left.

Below we have numbered single-line equations, the most common kind:
\begin{eqnarray}
\chi_+(p)\alt{\bf [}2|{\bf p}|(|{\bf p}|+p_z){\bf ]}^{-1/2}
\left(
\begin{array}{c}
|{\bf p}|+p_z\\
px+ip_y
\end{array}\right)\;,
\\
\left\{%
 \openone234567890abc123\alpha\beta\gamma\delta1234556\alpha\beta
 \frac{1\sum^{a}_{b}}{A^2}%
\right\}%
\label{eq:one}.
\end{eqnarray}
Note the open one in Eq.~(\ref{eq:one}).

Not all numbered equations will fit within a narrow column this
way. The equation number will move down automatically if it cannot fit
on the same line with a one-line equation:
\begin{equation}
\left\{
 ab12345678abc123456abcdef\alpha\beta\gamma\delta1234556\alpha\beta
 \frac{1\sum^{a}_{b}}{A^2}%
\right\}.
\end{equation}

When the \verb+\label{#1}+ command is used [cf. input for
Eq.~(\ref{eq:one})], the equation can be referred to in text without
knowing the equation number that \TeX\ will assign to it. Just
use \verb+\ref{#1}+, where \verb+#1+ is the same name that used in
the \verb+\label{#1}+ command.

Unnumbered single-line equations can be typeset
using the \verb+\[+, \verb+\]+ format:
\[g^+g^+ \rightarrow g^+g^+g^+g^+ \dots ~,~~q^+q^+\rightarrow
q^+g^+g^+ \dots ~. \]

\subsection{Multiline equations}

Multiline equations are obtained by using the \verb+eqnarray+
environment.  Use the \verb+\nonumber+ command at the end of each line
to avoid assigning a number:
\begin{eqnarray}
{\cal M}=&&ig_Z^2(4E_1E_2)^{1/2}(l_i^2)^{-1}
\delta_{\sigma_1,-\sigma_2}
(g_{\sigma_2}^e)^2\chi_{-\sigma_2}(p_2)\nonumber\\
&&\times
[\epsilon_jl_i\epsilon_i]_{\sigma_1}\chi_{\sigma_1}(p_1),
\end{eqnarray}
\begin{eqnarray}
\sum \vert M^{\text{viol}}_g \vert ^2&=&g^{2n-4}_S(Q^2)~N^{n-2}
        (N^2-1)\nonumber \\
 & &\times \left( \sum_{i<j}\right)
  \sum_{\text{perm}}
 \frac{1}{S_{12}}
 \frac{1}{S_{12}}
 \sum_\tau c^f_\tau~.
\end{eqnarray}
\textbf{Note:} Do not use \verb+\label{#1}+ on a line of a multiline
equation if \verb+\nonumber+ is also used on that line. Incorrect
cross-referencing will result. Notice the use \verb+\text{#1}+ for
using a Roman font within a math environment.

To set a multiline equation without \emph{any} equation
numbers, use the \verb+\begin{eqnarray*}+,
\verb+\end{eqnarray*}+ format:
\begin{eqnarray*}
\sum \vert M^{\text{viol}}_g \vert ^2&=&g^{2n-4}_S(Q^2)~N^{n-2}
        (N^2-1)\\
 & &\times \left( \sum_{i<j}\right)
 \left(
  \sum_{\text{perm}}\frac{1}{S_{12}S_{23}S_{n1}}
 \right)
 \frac{1}{S_{12}}~.
\end{eqnarray*}
To obtain numbers not normally produced by the automatic numbering,
use the \verb+\tag{#1}+ command, where \verb+#1+ is the desired
equation number. For example, to get an equation number of
(\ref{eq:mynum}),
\begin{equation}
g^+g^+ \rightarrow g^+g^+g^+g^+ \dots ~,~~q^+q^+\rightarrow
q^+g^+g^+ \dots ~. \tag{2.6$'$}\label{eq:mynum}
\end{equation}

A few notes on \verb=\tag{#1}=. \verb+\tag{#1}+ requires
\texttt{amsmath}. The \verb+\tag{#1}+ must come before the
\verb+\label{#1}+, if any. The numbering set with \verb+\tag{#1}+ is
\textit{transparent} to the automatic numbering in REV\TeX{};
therefore, the number must be known ahead of time, and it must be
manually adjusted if other equations are added. \verb+\tag{#1}+ works
with both single-line and multiline equations. \verb+\tag{#1}+ should
only be used in exceptional case - do not use it to number all
equations in a paper.

Enclosing single-line and multiline equations in
\verb+\begin{subequations}+ and \verb+\end{subequations}+ will produce
a set of equations that are ``numbered'' with letters, as shown in
Eqs.~(\ref{subeq:1}) and (\ref{subeq:2}) below:
\begin{subequations}
\label{eq:whole}
\begin{equation}
\left\{
 abc123456abcdef\alpha\beta\gamma\delta1234556\alpha\beta
 \frac{1\sum^{a}_{b}}{A^2}
\right\},\label{subeq:1}
\end{equation}
\begin{eqnarray}
{\cal M}=&&ig_Z^2(4E_1E_2)^{1/2}(l_i^2)^{-1}
(g_{\sigma_2}^e)^2\chi_{-\sigma_2}(p_2)\nonumber\\
&&\times
[\epsilon_i]_{\sigma_1}\chi_{\sigma_1}(p_1).\label{subeq:2}
\end{eqnarray}
\end{subequations}
Putting a \verb+\label{#1}+ command right after the
\verb+\begin{subequations}+, allows one to
reference all the equations in a subequations environment. For
example, the equations in the preceding subequations environment were
Eqs.~(\ref{eq:whole}).

\subsubsection{Wide equations}
The equation that follows is set in a wide format, i.e., it spans
across the full page. The wide format is reserved for long equations
that cannot be easily broken into four lines or less:
\begin{widetext}
\begin{equation}
{\cal R}^{(\text{d})}=
 g_{\sigma_2}^e
 \left(
   \frac{[\Gamma^Z(3,21)]_{\sigma_1}}{Q_{12}^2-M_W^2}
  +\frac{[\Gamma^Z(13,2)]_{\sigma_1}}{Q_{13}^2-M_W^2}
 \right)
 + x_WQ_e
 \left(
   \frac{[\Gamma^\gamma(3,21)]_{\sigma_1}}{Q_{12}^2-M_W^2}
  +\frac{[\Gamma^\gamma(13,2)]_{\sigma_1}}{Q_{13}^2-M_W^2}
 \right)\;. \label{eq:wideeq}
\end{equation}
\end{widetext}
This is typed to show the output is in wide format.
(Since there is no input line between \verb+\equation+ and
this paragraph, there is no paragraph indent for this paragraph.)

\end{document}
%
% ****** End of file aipsamp.tex ******
